\chapter{On the semantical differences between Java and C}

\section{Types}

\subsection{Primitive data types}
Java defines eight primitive data types: \javainline{boolean}, 
\javainline{char}, \javainline{byte}, \javainline{short}, \javainline{int},
\javainline{long}, \javainline{float}, \javainline{double}. All integral values 
are signed.

C defines one boolean type: \cinline{\_Bool}, five standard signed integer types: 
\cinline{signed char}, \cinline{short int}, \cinline{int}, \cinline{long int},
\cinline{long long int}, and three standard floating point types: \cinline{float},
\cinline{double}, and \cinline{long double} \cite[p.~40]{iso_c_standard}. The 
downside of the integral primitives is that their size is not exactly specified, 
and is thus implementation specific. For example, the \cinline{int} primtive can
be 32- or 64-bit, depending on the implementation. Luckily, CBMC defines four
exact size integral values: \cinline{\_\_int8}, \cinline{\_\_int16}, 
\cinline{\_\_int32}, and \cinline{\_\_int64} \cite[p.~39]{cprover_manual}.

We can define an exact mapping of the primitve data types of Java, to those of
C and CBMC. This mapping can be found in \ref{table:primitve_data_type_conversions}.

\begin{table}[H]
    \centering
    \caption{Mapping of the primitive data types.}
    \label{table:primitve_data_type_conversions}
    \begin{tabular}{lll}
    \hline
    \textbf{Type}        & \textbf{Description}                 & \textbf{C equivalent} \\ \hline
    \javainline{boolean} & true or false                        & \cinline{\_Bool}      \\
    \javainline{char}    & 16-bit Unicode value                 &                       \\
    \javainline{byte}    & 8-bit signed integral value          & \cinline{\_\_int8}    \\
    \javainline{short}   & 16-bit signed integral value         & \cinline{\_\_int16}   \\
    \javainline{int}     & 32-bit signed integral value         & \cinline{\_\_int32}   \\
    \javainline{long}    & 64-bit signed integral value         & \cinline{\_\_int64}   \\
    \javainline{float}   & IEEE 754 32-bit floating point value & \cinline{float}       \\
    \javainline{double}  & IEEE 754 64-bit floating point value & \cinline{double}      \\ \hline
    \end{tabular}
\end{table}

\subsection{Arrays}

\subsection{Strings}

\subsection{Classes}

\subsection{Interfaces}

\subsection{Enum}

\section{Expressions}

\subsection{Literals}

\subsection{Operators}

\subsection{Assignment}

\section{Namespaces}