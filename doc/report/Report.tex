\documentclass[a4paper]{book}

\usepackage[a4paper,top=3cm,bottom=2cm,left=3cm,right=3cm,marginparwidth=1.75cm]{geometry}
\usepackage{minted}
\usepackage{scrextend}
\usepackage{float}
\usepackage{hyperref}
\usepackage[page]{appendix}

\begin{document}

\title{A Bounded Verification Tool for Java}
\author{Stefan Koppier}
\maketitle

\chapter*{Introduction}

\tableofcontents

\chapter{CProver}

\section{Properties}

\begin{labeling}{arithmetic over- and underflow\quad}
    \item [array bounds] test
    \item [pointer] test
    \item [division by zero] test
    \item [arithmetic over- and underflow] test
    \item [shift greater than bit-width] test
    \item [floating-point for +/-Inf] test
    \item [floating-point for NaN] test
    \item [user assertions] test
\end{labeling}

\chapter{Translation of Java to C}

\section{Primitive data types}

\begin{table}[H]
    \centering
    \label{table:primitve_data_type_conversions}
    \caption{Equivalence of primitive Java data types.}
    \begin{tabular}{lll}
    \hline
    \textbf{Type} & \textbf{Description}                 & \textbf{C equivalent} \\ \hline
    boolean       & true or false                        & \_Bool                \\
    char          & 16-bit Unicode value                 &                       \\
    byte          & 8-bit signed integral value          & \_\_int8              \\
    short         & 16-bit signed integral value         & \_\_int16             \\
    int           & 32-bit signed integral value         & \_\_int32             \\
    long          & 64-bit integral value                & \_\_int64             \\
    float         & IEEE 754 64-bit floating point value & float                 \\
    double        & IEEE 754 32-bit floating point value & double                \\ \hline
    \end{tabular}
\end{table}

\bibliographystyle{plain}
\bibliography{bibliography.bib} 

\end{document}