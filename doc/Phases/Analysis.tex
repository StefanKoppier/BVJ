\chapter{Analysis}

%\subsection{Control Flow Analysis}
%We Control Flow Analysis (CFA) is the first step in the analysis process. We 
%take the abstract syntax tree of the input program and transform it into the 
%control flow graph (CFG).
%
%\begin{definition}
%The Control Flow Graph (CFG) of a program $P$ is a directed graph $G_P=(V,E)$
%where
%
%\begin{itemize}
%    \item $V$ is the set of nodes. Each node is one of either: the entry point 
%    of a method, the exit point of a method, a method call, or a basic block.
%
%    \item $E$ is the set of edges. Each edge is one of either: an interprocedural
%    edge, an intraprocedural edge, or a conditional edge.
%\end{itemize}
%\end{definition}
%
%\subsubsection{Intraprocedural Control Flow Analysis}
%The intraprocedural Control Flow Analysis, that is, the Control Flow within a
%single method, is constructed as follows:
%
%For every statement, a basic block is inserted, containing that statement.
%
%For the compound statements, we insert edges using the following approach:
%
%\begin{description}
%    \item[\javainline{while}]
%    \item[\javainline{if else}] 
%    \item[\javainline{switch}] 
%\end{description}
%
%
%\subsection{Reachability Analysis}
%Given a Control Flow Graph, a depth-first search is performed on the node $v \in V$ 
%that corresponds to the entry point of the method to be verified. The subgraph 
%containing only the nodes that are reached from $v$ is constructed.