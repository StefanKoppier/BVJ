\documentclass[a4paper]{book}

\usepackage[a4paper,top=3cm,bottom=2cm,left=3cm,right=3cm,marginparwidth=1.75cm]{geometry}
\usepackage{listings}
\usepackage{scrextend}
\usepackage[inline]{enumitem}
\usepackage{multicol}
\usepackage{float}
\usepackage{amsmath}
\usepackage{hyperref}
\usepackage{todonotes}
\usepackage{etoolbox}
\usepackage{tikz}
\usepackage{parskip}
\usepackage[page]{appendix}

\usetikzlibrary{shapes,arrows,positioning}

\newcommand\abstractname{Abstract}  %%% here
\makeatletter
\if@titlepage
  \newenvironment{abstract}{%
      \titlepage
      \null\vfil
      \@beginparpenalty\@lowpenalty
      \begin{center}%
        \bfseries \abstractname
        \@endparpenalty\@M
      \end{center}}%
     {\par\vfil\null\endtitlepage}
\else
  \newenvironment{abstract}{%
      \if@twocolumn
        \section*{\abstractname}%
      \else
        \small
        \begin{center}%
          {\bfseries \abstractname\vspace{-.5em}\vspace{\z@}}%
        \end{center}%
        \quotation
      \fi}
      {\if@twocolumn\else\endquotation\fi}
\fi
\makeatother

\newcommand{\javainline}[1]{\lstinline[language=Java
                                      ,basicstyle=\ttfamily
                                      ,showstringspaces=false]{#1}}

\newcommand{\cinline}[1]{\lstinline[language=C
                                   ,basicstyle=\ttfamily]{#1}}

\newcommand{\haskellinline}[1]{\lstinline[language=Haskell
                                         ,basicstyle=\ttfamily]{#1}}

\newcommand{\command}[1]{\lstinline[basicstyle=\ttfamily]{#1}}

\makeatletter
\AtEndEnvironment{C}{\xdef\xlang{Language: C}}
\AfterEndEnvironment{C}{\begin{flushright}\vspace{-2mm}\xlang\end{flushright}}
\makeatother

\makeatletter
\AtEndEnvironment{Java}{\xdef\xlang{Language: Java}}
\AfterEndEnvironment{Java}{\begin{flushright}\vspace{-2mm}\xlang\end{flushright}}
\makeatother

\lstnewenvironment{Haskell}[2]{
    \lstset{
        language=java
      , basicstyle=\ttfamily
      , label=#1
      , caption=#2
      , mathescape=true
      , showstringspaces=false
    }}{}
    
\lstnewenvironment{Java}[2]{
    \lstset{
        language=java
      , basicstyle=\ttfamily
      , frame=single
      , label=#1
      , caption=#2
      , mathescape=true
      , showstringspaces=false
    }}{}
    
\lstnewenvironment{C}[2]{
    \lstset{
          language=C
        , basicstyle=\ttfamily
        , frame=single
        , label=#1
        , caption=#2
    }}{}
     
\newtheorem{definition}{Definition}

\begin{document}

\title{A Bounded Verification Tool for Java source code}
\author{Stefan Koppier, \\ 6002978}
\maketitle

\begin{abstract}
We present a bounded model checking tool for a subset of the Java language. The 
tool allows the verification of user-defined assertions, and that runtime exceptions
never occur. We have built our tool on top of JBMC, a bounded model checking tool 
for Java bytecode. In this document we give a short description of JBMC, then
describe the architecture of the tool and finally conclude with results and possible
future work. The first results show that this tool successfully verifies the absence
and presence of bugs in multiple programs, of which merge sort is one example.
\end{abstract}

\chapter*{Introduction}
We present a tool that allows bounded model checking of Java source code. The
tool works as a layer over JBMC, a bounded verification tool for Java bytecode, 
developed by Lucas Cordeiro et al. \cite{ckkst2018}. JBMC is developed using 
the CPROVER framework, which also drives the industrial strength bounded model 
checking tool CBMC \cite{ckl2004}.

The development of this tool was inspired by CRUST. \cite{toman2015crust}. CRUST 
is a bounded model checking tool to find memory problems in Rust source code. The
goal of this project was to provide similar infrastructure as CRUST. That is: develop
a basis to do further bounded model checking research of Java source code.

The advantage of using this tool, over using JBMC itself is that we provide an 
interface for Java source code, instead of Java bytecode. In terms, this allows for
higher-level reasoning about the program path generation and verification process.

The tool is hosted on GitHub\footnote{\url{https://github.com/StefanKoppier/BVJ}}.

\section*{Structure of this document}
The document contains four chapters and one appendix. In chapter 
\ref{chap:jbmc}, we will give an introduction of JBMC and give an overview of 
similar tools. Chapter 
\ref{chap:phases} will explain the architecture of the tool, and explain each
individual phase and what it tries to achieve. Chapter \ref{chap:usage} will
give insight on how to use the tool. We will conclude in chapter \ref{chap:conclusion},
where we also discuss possible future work. Appendix \ref{app:examples} show some
examples and their results.

\section*{A change of course*}
Out initial approach was to compile the Java programs to program paths in C. This
turned out to be too much work, so we changed our approach to generate program 
paths in Java.

\subsection*{A first approach: compiling to C}
We first envisioned to compile all Java program paths to program paths in C. 
We could then perform bounded model checking using CBMC \cite{ckl2004}. 

The advantages we foresaw compiling to C were that CBMC is a more mature tool than
JBMC. But after a while, the work required to translate the Java semantics to 
equivalent C semantics turned out to be overwhelming for the time we had available. 
More specifically, translating the exception handling system of Java required us 
to define a semantically equivalent exception handling system in C, as the 
language has no native support for exceptions. Besides the exception handling system, 
compiling to C had more disadvantages. It required array initialization to contain 
loop structures in the compiled code, something we wished to avoid. For example, 
see the Java input program and compiled C program in listings \ref{listing:java_array} 
and \ref{listing:c_array}. The array is initialized using constants, but suppose it 
is initialized with some computationally heavy function, and the initialization 
is inside some loop. This will generate many program paths which are hard to verify.

\begin{Java}{listing:java_array}{Java program containing array initialization.}
class Main {
    public static void main() {
        int[] array = new int[] { 1, 2 };
    }
}
\end{Java}

\begin{C}{listing:c_array}{Compiled C program containing the array initialization.}
#include <stdlib.h>
struct Int_Array {
    __int32 * elements; __int32 length;
};
struct Int_Array * new_Array$0()
{
    __int32 initial[2] = { 1, 2 };
    struct Int_Array * this = malloc(sizeof(struct Int_Array));
    {
        this->length = 2;
        this->elements = calloc(2, sizeof(__int32));
        for (__int32 i0 = 0; i0 < 2; ++i0)
            this->elements[i0] = initial[i0];
    }
    return this;
}
void main()
{
    struct Int_Array * array = new_Array$0();
}
\end{C}

We finalized the C version of the tool as version v0.1, which can be found on 
GitHub\footnote{\url{https://github.com/StefanKoppier/BVJ/releases/tag/v0.1}}.
The final C version has support for a similar Java subset as the Java version of
the tool, excluding the exception handling system.

\subsection*{A second approach: compiling to Java}
The second approach was to compile our Java program to Java program paths, and
verify these using JBMC \cite{ckkst2018}. 

The main advantage of this approach is that we are semantically much closer to
our input program, only a few modifications to the compiled program paths are
needed. The main initial advantage is that the exception handling system is 
supported, which doesn't require us to define our own. Other advantages are that 
this approach is more extensible; it is easier to add new features, for it not 
requires a equivalent semantical C definition of the new features.

A disadvantage of the Java version of the tool is that JBMC is less mature than
CBMC. But the upside is that JBMC received funding and thus is in active 
development.

The rest of this document describes the Java version of the tool.

\tableofcontents

\chapter{JBMC: a bounded model checking tool for Java bytecode} \label{chap:jbmc}
The tool is built upon JBMC, an actively developed bounded model checker for 
verifying Java bytecode \cite{ckkst2018}. JBMC is built upon the CProver framework,
which is the basis of CBMC: an industrial strength bounded model checker for C 
and C++ programs. JBMC works by translating Java bytecode into what they call a
GOTO program; a control flow graph representation of the program. This GOTO 
program is symbolically executed to generate a bit-vector formula. This bit-vector
formula is verified by a SAT or SMT solver, which by default is MiniSat 
\cite{een2003extensible}.

JBMC has support for checking that runtime exceptions do not occur and that 
user-defined assertions are valid. Assertions are natively supported in Java 
and can be written using the \javainline{assert e;} and \javainline{assert e : "foo";} 
statements. Assumptions can be written using a method call: \javainline{CProver.assume(e);}.

The team of JBMC actively maintains a model of the Java Class Library (JCL), 
the standard library included with the Java language. Currently, the main support
of the model is focussed on including the Exception and String types. The model
is maintained on GitHub\footnote{https://github.com/diffblue/java-models-library}.

\section{Future work of JBMC}
The team working on JBMC claim in their tool paper \cite{ckkst2018} that there
currently is no support for the Java Native Interface, reflection, generics, 
lambda expressions, or multi-threaded programs. Their current plan is to extend 
JBMC to support the latter three.

\section{A small survey of formal verification tools for Java}
Besides JBMC, there exist several tools that aim to achieve a similar result.

JayHorn is a framework for verifying Java bytecode \cite{kahsai2016jayhorn}. It 
works by transforming the input to Horn clauses and solving these clauses using 
a theorem prover. JayHorn claims to have soundness as its main focus. Although 
they claim soundness as their main focus, the development has not yet reached a
point that this is the case. They do not have a stable release thus far that is
sound. The project has recent activity in their GitHub codebase, so it seems to
be under active development. On their GitHub 
page\footnote{https://github.com/jayhorn/jayhorn}, they state that the following
features are not fully sound:

\begin{itemize}
    \item JNI, implicit method invocations (finalizers, class initializers, Thread.<init>, etc.)
    \item integer overflow
    \item exceptions and flow related to that
    \item reflection API (e.g., Method.invoke(), Class.newInstance)
    \item invokedynamic
    \item code generation at runtime, dynamic loading
    \item different class loaders
    \item key native methods (Object.run, Object.doPrivileged)
\end{itemize}

Java Pathfinder (jpf) is a system to fomrally verify Java bytecode \cite{havelund2000model}.
JPF focusses on verifying the absence of deadlocks and user-defined assertions.
JPF works by constructing a Promela program, which can be fed into SPIN to be
verified for correctness. JPF was developed at the NASA Ames Research Center and
was made open source in 2015 under the Apache 2.0 license. The project is hosted
on GitHub\footnote{https://github.com/javapathfinder/}.

Clausio Giovanni Demartini et al. \cite{Demartini1998ModelingAV} describe a 
similar approach as jpf, the Java2Spin translator. Is also verifies the absence 
of deadline, by translating a Java programs into a Promela program. Although 
successful experimental results, I couldn't find a tool based on this paper under 
active development. 



\chapter{Phases of the Tool}

\section{Lexing and Parsing}

\chapter{Analysis}

%\subsection{Control Flow Analysis}
%We Control Flow Analysis (CFA) is the first step in the analysis process. We 
%take the abstract syntax tree of the input program and transform it into the 
%control flow graph (CFG).
%
%\begin{definition}
%The Control Flow Graph (CFG) of a program $P$ is a directed graph $G_P=(V,E)$
%where
%
%\begin{itemize}
%    \item $V$ is the set of nodes. Each node is one of either: the entry point 
%    of a method, the exit point of a method, a method call, or a basic block.
%
%    \item $E$ is the set of edges. Each edge is one of either: an interprocedural
%    edge, an intraprocedural edge, or a conditional edge.
%\end{itemize}
%\end{definition}
%
%\subsubsection{Intraprocedural Control Flow Analysis}
%The intraprocedural Control Flow Analysis, that is, the Control Flow within a
%single method, is constructed as follows:
%
%For every statement, a basic block is inserted, containing that statement.
%
%For the compound statements, we insert edges using the following approach:
%
%\begin{description}
%    \item[\javainline{while}]
%    \item[\javainline{if else}] 
%    \item[\javainline{switch}] 
%\end{description}
%
%
%\subsection{Reachability Analysis}
%Given a Control Flow Graph, a depth-first search is performed on the node $v \in V$ 
%that corresponds to the entry point of the method to be verified. The subgraph 
%containing only the nodes that are reached from $v$ is constructed.


\chapter{How to use the tool} \label{chap:usage}
The tool can be used in two ways: using GHCi and by running the excutable. The tool
can be build using the command \command{make repl}, which also starts GHCi.

When GHCi is used, there are 3 functions which provide an easy to use interface
to verify: \haskellinline{verify}, \haskellinline{verifyWithMaximumDepth}, and
\haskellinline{verifyFunctionWithMaximumDepth}. There is an \haskellinline{arguments}
constant defined which allow for the tweaking of the arguments for the verification
process.

When running the executable, the arguments in listing \ref{listing:arguments} 
can be used to tweak the argument for the verification process. For example,
the command \command{bvj "examples/Test.java" -c -k10 -t4} verifies the file
\command{examples/Test.java} printing minimal information, with a maximum
program path length of 10 and uses 4 threads.

\begin{lstlisting}[basicstyle=\small\ttfamily, caption={Argument for the executable}, label={listing:arguments}]
Parameters:
File    Path to the file to be verified.

Flags:
-c           --compact                        Display less information.
-r           --remove                         Remove the output files.
-f[FUNCTION] --function[=FUNCTION]            Verify the function.
-k[DEPTH]    --depth[=DEPTH]                  Maximum program path generation depth.
-t[THREADS]  --threads[=THREADS]              Number of threads.
-u[UNWIND]   --unwind[=UNWIND]                Maximum loop unwinding in JBMC.
             --verification-depth[=VER-DEPTH] Maximum depth in JBMC.
\end{lstlisting}

\section{Prerequisites}
Using the tool requires the following:

\begin{enumerate}
    \item The Java development kit installed, ensure that \command{javac} and
    \command{jar} can be found in the PATH variable;
    \item the JBMC tool, ensure that that \command{jbmc} can be found in the PATH 
    variable;
    \item the following Haskell libraries:
    \begin{multicols}{3}
        \begin{itemize}
            \item \command{fgl 5.7.0.0}
            \item \command{directory 1.3.0.2}
            \item \command{terminal-progress-bar}
            \item \command{command}
            \item \command{filepath}
            \item \command{dates}
            \item \command{pretty}
            \item \command{xeno}
            \item \command{utf8-string}
            \item \command{language-java}
            \item \command{transformers}
            \item \command{containers}
            \item \command{parallel-io}
            \item \command{split}
            \item \command{simple-get-opt}
        \end{itemize}
    \end{multicols}
\end{enumerate}


\chapter{Conclusion} \label{chap:conclusion}
Concluding, we presented a bounded model checking tool which covers a subset of
the Java language. The examples in the appendix show that the tool can correctly 
detect errors in algorithms, like merge sort. Making this a feasible approach to
formally verify Java source code, instead of Java bytecode, which other bounded
model checking tools focus on.

\section{Possible future work}
\subsection*{Extend the subset of the Java language}
Currently, the tool only allows for the verification of a subset of the Java
language. This can be extended to allow for more Java programs to be verified.

One specific example is allowing \javainline{break} and \javainline{continue}
statements inside a \javainline{try}, \javainline{catch}, or \javainline{finally}
block inside a loop.

\subsection*{Add support for multi-threaded programs}
The developers of JBMC claim to have multi-threading support as one of their top
priorities. When JBMC supports multi-threading, the tool can be extended to support
multi-threading as well.

\bibliographystyle{plain}
\bibliography{bibliography.bib}

\begin{appendices}
    \chapter{Examples} \label{app:examples}

\section{The Fibonacci sequence}
Let us have a look at the program in listing \ref{listing:example_fibonacci}. This
program computes the second fibonacci number, and asserts if it is equal to 1. This
program generates the control flow graph shown in figure \ref{fig:cfg_fib}.

\begin{Java}{listing:example_fibonacci}{\text{Program verifying the fib method.}}
class Main {
    public static void main(String[] argv) {
        assert fib(2) == 1 : "fib(2) != 1";
    }

    public static int fib(int x) {
        if (x == 0) {
            return 0;
        } else if (x == 1) {
            return 1;
        } else {
            return fib(x - 1) + fib(x - 2);
        }
    }
}
\end{Java}

When we verify this program using $k=10$, we get the following five program 
paths, of which the shortest is:

\begin{lstlisting}[basicstyle=\small\ttfamily]
[CProver.assume(x==0); { return 0; } assert Main_fib0(2)==1 : "fib(2) != 1";]
\end{lstlisting}

The tool shows that all five program paths are correct. 

Modifying the \javainline{assert fib(2) == 1 : "fib(2) != 1";} to be
\javainline{assert fib(2) == 100 : "fib(2) != 100";} and running the tool again
shows that the assertion does not hold.

\begin{figure}[H]
    \caption{Control flow graph of listing \ref{listing:example_fibonacci}.}
    \label{fig:cfg_fib}
    \centering
    \begin{tikzpicture}[node/.style = {draw, circle}, node distance=1.23cm]
        \node(1)  [label=left:{entry of 'Main.main'}]        {1};
        \node(2)  [below=of 1,label=left:{;}]                           {2};
        \node(3)  [below=of 2,label=left:{call of 'Main.fib'}]          {3};
        \node(7)  [right=of 3,label={entry of 'Main.fib'}]         {7};
        \node(8)  [below=of 7,label=left:{;}]                           {8};
        \node(9)  [below=of 8,label=left:{;}]                           {9};
        \node(10) [below=of 9,label=left:{if (x == 0)}]                 {10};
        \node(11) [below left=of 10,label=left:{return 0;}]                   {11};
        \node(12) [below right=of 10,label=right:{;}]                           {12};
        \node(13) [below=of 12,label=left:{if (x == 1)}]                 {13};
        \node(14) [below left=of 13,xshift=-1cm,label=left:{return 1;}]                   {14};
        \node(15) [below right=of 13,label=left:{call of 'Main.fib'}]          {15};
        \node(16) [below=of 15,label=left:{call of 'Main.fib'}]          {16};
        \node(17) [below=of 16,label={return fib(x-1) + fib(x-2);}] {17};
        \node(18) [below=of 17,label=right:{;}]                           {18};
        \node(19) [below=of 18,label=right:{;}]                           {19};
        \node(20) [below=of 19,label=right:{;}]                           {20};
        \node(21) [below=of 20,label=below:{exit of 'Main.fib'}]          {21};
        \node(4)  [left=of 21,xshift=-0.25cm,label=above:{assert fib(2) == 1}]          {4};
        \node(5)  [left=of 4,xshift=-0.25cm,label=above:{;}]                           {5};
        \node(6)  [left=of 5,xshift=-0.25cm,label=above:{exit of 'Main.main}]          {6};

        \draw[->] (1) -- (2);
        \draw[->] (2) -- (3);
        \draw[->] (3) -- (7);
        \draw[->] (4) -- (5);
        \draw[->] (5) -- (6);
        \draw[->] (7) -- (8);
        \draw[->] (8) -- (9);
        \draw[->] (9) -- (10);
        \draw[->] (10) -- (11);
        \draw[->] (10) -- (12);
        \draw[->] (11.south) to [bend right=90] (19.west);
        \draw[->] (12) -- (13);
        \draw[->] (13) -- (14);
        \draw[->] (13) -- (15);
        \draw[->] (14.south) to [bend right=90] (18.west);
        \draw[->] (15.east) to [bend right] (7.east);
        \draw[->] (16.east) to [bend right] (7.east);
        \draw[->] (17) -- (18);
        \draw[->] (18) -- (19);
        \draw[->] (19) -- (20);
        \draw[->] (20) -- (21);
        \draw[->] (21) -- (4);
        \draw[->] (21.east) to [bend right=90] (16.east);
        \draw[->] (21.east) to [bend right] (17.east);
    \end{tikzpicture}
\end{figure}

\section{Merge sort}
Let us have a look at the program in listing \ref{listing:example_sort}. This 
program defined an array \javainline{elems} which is passed to the \javainline{sort}
method, which is an implementation of merge sort.

The tool correctly verifies the assertion that the array is sorted.

\begin{Java}{listing:example_sort}{Program verifying the sort method.}
class Main {
    public static void main(String[] argv) {
        int[] elems = new int[] { 2, 1 };
        sort(elems, 0, elems.length - 1);
        assert (elems[0] == 1) && (elems[1] == 2);
    }

    public static int partition(int[] array, int low, int high)
    {
        int pivot = array[high];
        int i = low - 1;
        int temp;

        for (int j = low; j < high; j++) {
            if (array[j] <= pivot)
            {
                i++;
                temp = array[i];
                array[i] = array[j];
                array[j] = temp;
            }
        }

        temp = array[i+1];
        array[i+1] = array[high];
        array[high] = temp;

        return i + 1;
    }

    public static void sort(int[] array, int low, int high) {
        if (low < high) {
            int pi = partition(array, low, high);
            sort(array, low, pi - 1);
            sort(array, pi + 1, high);
        }
    }
}
\end{Java}
\end{appendices}

\end{document}