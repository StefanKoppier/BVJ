\documentclass[a4paper]{book}

\usepackage[a4paper,top=3cm,bottom=2cm,left=3cm,right=3cm,marginparwidth=1.75cm]{geometry}
\usepackage{listings}
\usepackage{scrextend}
\usepackage[inline]{enumitem}
\usepackage{multicol}
\usepackage{float}
\usepackage{amsmath}
\usepackage{hyperref}
\usepackage{todonotes}
\usepackage{etoolbox}
\usepackage{tikz}
\usepackage{parskip}
\usepackage[page]{appendix}

\usetikzlibrary{shapes,arrows,positioning}

\newcommand\abstractname{Abstract}  %%% here
\makeatletter
\if@titlepage
  \newenvironment{abstract}{%
      \titlepage
      \null\vfil
      \@beginparpenalty\@lowpenalty
      \begin{center}%
        \bfseries \abstractname
        \@endparpenalty\@M
      \end{center}}%
     {\par\vfil\null\endtitlepage}
\else
  \newenvironment{abstract}{%
      \if@twocolumn
        \section*{\abstractname}%
      \else
        \small
        \begin{center}%
          {\bfseries \abstractname\vspace{-.5em}\vspace{\z@}}%
        \end{center}%
        \quotation
      \fi}
      {\if@twocolumn\else\endquotation\fi}
\fi
\makeatother

\newcommand{\javainline}[1]{\lstinline[language=Java
                                      ,basicstyle=\ttfamily
                                      ,showstringspaces=false]{#1}}

\newcommand{\cinline}[1]{\lstinline[language=C
                                   ,basicstyle=\ttfamily]{#1}}

\newcommand{\haskellinline}[1]{\lstinline[language=Haskell
                                         ,basicstyle=\ttfamily]{#1}}

\newcommand{\command}[1]{\lstinline[basicstyle=\ttfamily]{#1}}

\makeatletter
\AtEndEnvironment{C}{\xdef\xlang{Language: C}}
\AfterEndEnvironment{C}{\begin{flushright}\vspace{-2mm}\xlang\end{flushright}}
\makeatother

\makeatletter
\AtEndEnvironment{Java}{\xdef\xlang{Language: Java}}
\AfterEndEnvironment{Java}{\begin{flushright}\vspace{-2mm}\xlang\end{flushright}}
\makeatother

\lstnewenvironment{Haskell}[2]{
    \lstset{
        language=java
      , basicstyle=\ttfamily
      , label=#1
      , caption=#2
      , mathescape=true
      , showstringspaces=false
    }}{}
    
\lstnewenvironment{Java}[2]{
    \lstset{
        language=java
      , basicstyle=\ttfamily
      , frame=single
      , label=#1
      , caption=#2
      , mathescape=true
      , showstringspaces=false
    }}{}
    
\lstnewenvironment{C}[2]{
    \lstset{
          language=C
        , basicstyle=\ttfamily
        , frame=single
        , label=#1
        , caption=#2
    }}{}
     
\newtheorem{definition}{Definition}

\begin{document}

\title{A Bounded Verification Tool for Java source code}
\author{Stefan Koppier, \\ 6002978}
\maketitle

\begin{abstract}
We present a bounded model checking tool for a subset of the Java language. The 
tool allows the verification of user-defined assertions, and that runtime exceptions
never occur. We have built our tool on top of JBMC, a bounded model checking tool 
for Java bytecode. In this document we give a short description of JBMC, then
describe the architecture of the tool and finally conclude with results and possible
future work. The first results show that this tool successfully verifies the absence
and presence of bugs in multiple programs, of which merge sort is one example.
\end{abstract}

\chapter*{Introduction}
We present a tool that allows bounded model checking of Java source code. The
tool works as a layer over JBMC, a bounded verification tool for Java bytecode, 
developed by Lucas Cordeiro et al. \cite{ckkst2018}. JBMC is developed using 
the CPROVER framework, which also drives the industrial strength bounded model 
checking tool CBMC \cite{ckl2004}.

The development of this tool was inspired by CRUST. \cite{toman2015crust}. CRUST 
is a bounded model checking tool to find memory problems in Rust source code. The
goal of this project was to provide similar infrastructure as CRUST. That is: develop
a basis to do further bounded model checking research of Java source code.

The advantage of using this tool, over using JBMC itself is that we provide an 
interface for Java source code, instead of Java bytecode. In terms, this allows for
higher-level reasoning about the program path generation and verification process.

The tool is hosted on GitHub\footnote{\url{https://github.com/StefanKoppier/BVJ}}.

\section*{Structure of this document}
The document contains four chapters and one appendix. In chapter 
\ref{chap:jbmc}, we will give an introduction of JBMC and give an overview of 
similar tools. Chapter 
\ref{chap:phases} will explain the architecture of the tool, and explain each
individual phase and what it tries to achieve. Chapter \ref{chap:usage} will
give insight on how to use the tool. We will conclude in chapter \ref{chap:conclusion},
where we also discuss possible future work. Appendix \ref{app:examples} show some
examples and their results.

\section*{A change of course*}
Out initial approach was to compile the Java programs to program paths in C. This
turned out to be too much work, so we changed our approach to generate program 
paths in Java.

\subsection*{A first approach: compiling to C}
We first envisioned to compile all Java program paths to program paths in C. 
We could then perform bounded model checking using CBMC \cite{ckl2004}. 

The advantages we foresaw compiling to C were that CBMC is a more mature tool than
JBMC. But after a while, the work required to translate the Java semantics to 
equivalent C semantics turned out to be overwhelming for the time we had available. 
More specifically, translating the exception handling system of Java required us 
to define a semantically equivalent exception handling system in C, as the 
language has no native support for exceptions. Besides the exception handling system, 
compiling to C had more disadvantages. It required array initialization to contain 
loop structures in the compiled code, something we wished to avoid. For example, 
see the Java input program and compiled C program in listings \ref{listing:java_array} 
and \ref{listing:c_array}. The array is initialized using constants, but suppose it 
is initialized with some computationally heavy function, and the initialization 
is inside some loop. This will generate many program paths which are hard to verify.

\begin{Java}{listing:java_array}{Java program containing array initialization.}
class Main {
    public static void main() {
        int[] array = new int[] { 1, 2 };
    }
}
\end{Java}

\begin{C}{listing:c_array}{Compiled C program containing the array initialization.}
#include <stdlib.h>
struct Int_Array {
    __int32 * elements; __int32 length;
};
struct Int_Array * new_Array$0()
{
    __int32 initial[2] = { 1, 2 };
    struct Int_Array * this = malloc(sizeof(struct Int_Array));
    {
        this->length = 2;
        this->elements = calloc(2, sizeof(__int32));
        for (__int32 i0 = 0; i0 < 2; ++i0)
            this->elements[i0] = initial[i0];
    }
    return this;
}
void main()
{
    struct Int_Array * array = new_Array$0();
}
\end{C}

We finalized the C version of the tool as version v0.1, which can be found on 
GitHub\footnote{\url{https://github.com/StefanKoppier/BVJ/releases/tag/v0.1}}.
The final C version has support for a similar Java subset as the Java version of
the tool, excluding the exception handling system.

\subsection*{A second approach: compiling to Java}
The second approach was to compile our Java program to Java program paths, and
verify these using JBMC \cite{ckkst2018}. 

The main advantage of this approach is that we are semantically much closer to
our input program, only a few modifications to the compiled program paths are
needed. The main initial advantage is that the exception handling system is 
supported, which doesn't require us to define our own. Other advantages are that 
this approach is more extensible; it is easier to add new features, for it not 
requires a equivalent semantical C definition of the new features.

A disadvantage of the Java version of the tool is that JBMC is less mature than
CBMC. But the upside is that JBMC received funding and thus is in active 
development.

The rest of this document describes the Java version of the tool.

\tableofcontents

\chapter{JBMC: a bounded model checking tool for Java bytecode} \label{chap:jbmc}
The tool is built upon JBMC, an actively developed bounded model checker for 
verifying Java bytecode \cite{ckkst2018}. JBMC is built upon the CProver framework,
which is the basis of CBMC: an industrial strength bounded model checker for C 
and C++ programs. JBMC works by translating Java bytecode into what they call a
GOTO program; a control flow graph representation of the program. This GOTO 
program is symbolically executed to generate a bit-vector formula. This bit-vector
formula is verified by a SAT or SMT solver, which by default is MiniSat 
\cite{een2003extensible}.

JBMC has support for checking that runtime exceptions do not occur and that 
user-defined assertions are valid. Assertions are natively supported in Java 
and can be written using the \javainline{assert e;} and \javainline{assert e : "foo";} 
statements. Assumptions can be written using a method call: \javainline{CProver.assume(e);}.

The team of JBMC actively maintains a model of the Java Class Library (JCL), 
the standard library included with the Java language. Currently, the main support
of the model is focussed on including the Exception and String types. The model
is maintained on GitHub\footnote{https://github.com/diffblue/java-models-library}.

\section{Future work of JBMC}
The team working on JBMC claim in their tool paper \cite{ckkst2018} that there
currently is no support for the Java Native Interface, reflection, generics, 
lambda expressions, or multi-threaded programs. Their current plan is to extend 
JBMC to support the latter three.

\section{A small survey of formal verification tools for Java}
Besides JBMC, there exist several tools that aim to achieve a similar result.

JayHorn is a framework for verifying Java bytecode \cite{kahsai2016jayhorn}. It 
works by transforming the input to Horn clauses and solving these clauses using 
a theorem prover. JayHorn claims to have soundness as its main focus. Although 
they claim soundness as their main focus, the development has not yet reached a
point that this is the case. They do not have a stable release thus far that is
sound. The project has recent activity in their GitHub codebase, so it seems to
be under active development. On their GitHub 
page\footnote{https://github.com/jayhorn/jayhorn}, they state that the following
features are not fully sound:

\begin{itemize}
    \item JNI, implicit method invocations (finalizers, class initializers, Thread.<init>, etc.)
    \item integer overflow
    \item exceptions and flow related to that
    \item reflection API (e.g., Method.invoke(), Class.newInstance)
    \item invokedynamic
    \item code generation at runtime, dynamic loading
    \item different class loaders
    \item key native methods (Object.run, Object.doPrivileged)
\end{itemize}

Java Pathfinder (jpf) is a system to fomrally verify Java bytecode \cite{havelund2000model}.
JPF focusses on verifying the absence of deadlocks and user-defined assertions.
JPF works by constructing a Promela program, which can be fed into SPIN to be
verified for correctness. JPF was developed at the NASA Ames Research Center and
was made open source in 2015 under the Apache 2.0 license. The project is hosted
on GitHub\footnote{https://github.com/javapathfinder/}.

Clausio Giovanni Demartini et al. \cite{Demartini1998ModelingAV} describe a 
similar approach as jpf, the Java2Spin translator. Is also verifies the absence 
of deadline, by translating a Java programs into a Promela program. Although 
successful experimental results, I couldn't find a tool based on this paper under 
active development. 



\chapter{The phases of the tool} \label{chap:phases}
The tool consists of five main phases: parsing, control flow analysis, 
linearization, compilation and verification. A global overview of all these phases
and their inputs and outputs can be found in figure \ref{fig:phases_overview}.

\begin{figure}[H]
    \caption{An overview of the phases of the tool.}
    \centering
    \label{fig:phases_overview}
    \begin{tikzpicture}[node distance=0.85cm and 0.2cm, align=center]
        \tikzset{
            phase/.style      = { rectangle
                                , rounded corners 
                                , minimum width=2.5cm
                                , text centered
                                , draw=black
                                }
        , intermediate/.style = { rectangle
                                , minimum width=2.5cm
                                }
        , dots/.style         = { rectangle
                                , minimum width=0.5cm
                                }
        , arrow/.style        = { very thick
                                , shorten >= 0.2cm
                                , shorten <= 0.2cm
                                }
        }

        \node(input)             [intermediate]                       {Java source file};
        \node(parser)            [phase,below=of input] {Parsing};
            \node() [right=of parser] {module \haskellinline{Parsing.Phase}};
        \node(ast)               [intermediate,below=of parser]   {Abstract Syntax Tree};
        \node(analyzer)          [phase,below=of ast]    {Control Flow\\Analysis};
            \node() [right=of analyzer] {module \haskellinline{Analysis.Phase}};
        \node(cfg)               [intermediate,below=of analyzer]    {Control Flow Graph};
        \node(linearizer)        [phase,below=of cfg]              {Linearization};
            \node() [right=of linearizer] {module \haskellinline{Linearization.Phase}};
        \node(path_dots)         [dots,below=of linearizer]      {$\cdots$};
        \node(path_1)            [intermediate,left=of path_dots] {$\text{Program Path}_1$};
        \node(path_n)            [intermediate,right=of path_dots] {$\text{Program Path}_n$};
        \node(compiler_dots)     [dots,below=of path_dots]      {$\cdots$};
        \node(compiler_1)        [phase,left=of compiler_dots] {Compilation};
        \node(compiler_n)        [phase,right=of compiler_dots] {Compilation};
            \node() [right=of compiler_n] {module \haskellinline{Compilation.Phase}};
        \node(bytecode_dots)     [dots,below=of compiler_dots]      {$\cdots$};
        \node(bytecode_1)        [intermediate,left=of bytecode_dots] {$\text{Java Bytecode}_1$};
        \node(bytecode_n)        [intermediate,right=of bytecode_dots] {$\text{Java Bytecode}_n$};
        \node(verification_dots) [dots,below=of bytecode_dots]      {$\cdots$};
        \node(verification_1)    [phase,left=of verification_dots] {Verification};
        \node(verification_n)    [phase,right=of verification_dots] {Verification};
            \node() [right=of verification_n] {module \haskellinline{Verification.Phase}};
        \node(results)           [intermediate,below of=verification_dots] {Verification Results};

        \draw[arrow,->] (input) -- (parser);
        \draw[arrow,->] (parser) -- (ast);
        \draw[arrow,->] (ast) -- (analyzer);
        \draw[arrow,->] (analyzer) -- (cfg);
        \draw[arrow,->] (cfg) -- (linearizer);
        \draw[arrow,->] (linearizer) -- (path_1);
        \draw[arrow,->] (linearizer) -- (path_n);
        \draw[arrow,->] (path_1) -- (compiler_1);
        \draw[arrow,->] (path_n) -- (compiler_n);
        \draw[arrow,->] (compiler_1) -- (bytecode_1);
        \draw[arrow,->] (compiler_n) -- (bytecode_n);
        \draw[arrow,->] (bytecode_1) -- (verification_1);
        \draw[arrow,->] (bytecode_n) -- (verification_n);
        \draw[arrow,->] (verification_1) -- (results);
        \draw[arrow,->] (verification_n) -- (results);
    \end{tikzpicture}
\end{figure}

All phases are chained into one phase, the Complete phase. This phase is defined
in the \haskellinline{Complete} module and is the main API of the tool.

\section{Lexing and Parsing}
Lexing and parsing is the first phase of the tool. It consists of two subphases:
the lexer and parser, and the syntax transformation. The complete phase takes a 
\haskellinline{String} as input, and transforms it to a \haskellinline{CompilationUnit'}
defined in the \haskellinline{Parsing.Syntax} module. The 
\haskellinline{CompilationUnit'} can be seen as the root node of the abstract
syntax tree.

The lexing and parsing is done using an intermediate abstract syntax tree, defined
in the library \href{http://hackage.haskell.org/package/language-java}{language-java}.
This library also contains the lexer and parser that is used within the tool.

The output of the parser of the 
\href{http://hackage.haskell.org/package/language-java}{language-java} library 
is fed into a syntax transformation subphase. This phase transforms the abstract 
syntax tree of the \href{http://hackage.haskell.org/package/language-java}{language-java} 
library into the abstract syntax tree defined in \haskellinline{Parsing.Syntax}.
Our abstract syntax tree is almost an one-to-one correspondence of the abstract
syntax tree defined in \href{http://hackage.haskell.org/package/language-java}{language-java}.

The definitions in \haskellinline{Parsing.Syntax} are contained in an attribute 
grammar file, which can be built using the Attribute Grammar System of Utrecht
University\footnote{\url{hackage.haskell.org/package/uuagc}}. This system allows
easy information retrieval of the abstract syntax tree, but at the cost that we
cannot use the abstract syntax tree defined in the 
\href{http://hackage.haskell.org/package/language-java}{language-java} library 
directly.

\subsection{The supported subset of Java}
Only a subset of the Java language is supported. There is support for a single
Java source file containing one or more class declarations. Interfaces and enum 
declarations are not supported. 

Classes may contain fields, methods and constructors. All modifiers on these 
levels are supported, including annotations. Classes may not contain generics, 
note that JBMC also does not have support for generics. Methods and constructors
may not be overloaded, there cannot exist two methods with the same name in the
same class.

Methods and constructors contain statements, of which we distinct two types. 
Compound statements, and basic statements. Compound statements are statements 
which itself contain statements. Basic statements are statements which do 
not contain statements. A constructor may not call its base constructor explicitly.
The compound statements that are supported are:

\begin{itemize}
    \item A block, introduced using curly braces;
    \item an \javainline{if else} statement;
    \item a \javainline{while} loop;
    \item a \javainline{try catch finally} statement;
    \item a \javainline{for} loop.
\end{itemize}

Note that \javainline{while}- and \javainline{for} loops can be labeled, e.g.
\javainline{l: while(g)}.

The basic statements that are are supported are:

\begin{multicols}{2}
    \begin{itemize}
        \item A variable declaration;
        \item a skip statement;
        \item an expression statement;
        \item an \javainline{assert} statement;
        \item a \javainline{break} statement;
        \item a \javainline{continue} statement;
        \item a \javainline{return} statement;
        \item a \javainline{throw} statement.
\end{itemize}
\end{multicols}

An assume statement can be written as \javainline{CProver.assume(e)}. The package
that contains the \javainline{CProver} class does not have to be included as this 
is done by the compiler.

There is one contextual issue: having a \javainline{break} or 
\javainline{continue} statement transferring the control flow from a 
\javainline{try}, \javainline{catch} or \javainline{finally} statement. For 
example, see listing \ref{listing:java_break_problem}. This code is semantically 
valid, but fails to compile correctly. 

\begin{Java}{listing:java_break_problem}{\text{Breaking from a catch, which is problematic.}}
void foo() {
    int x = 0;
    while (true) {
        try {
            x = x / x;
        } catch (ArithmeticException e) {
            break;
        }
    }
}
\end{Java}

All expression constructs are supported, except for three. There is no support for 
casting, the \javainline{instanceof} operator, and lambda expressions. Lambda 
expressions are unsupported by JBMC.


\section{Analysis}
The second phase of the tool is the Control Flow Analysis. It takes a 
\haskellinline{CompilationUnit'} as input and transforms it into a 
\haskellinline{CFG}, defined in the \haskellinline{Analysis.CFG} module.

The control flow graph is implemented using the inductive graph representation
of the \href{http://hackage.haskell.org/package/fgl}{fgl} library. The inductive
graph is a graph suitable for functional style programming. They are developed
by Martin Erwig \cite{erwig2001inductive}.

\begin{definition}
The Control Flow Graph (CFG) of a Java program $P$ is a directed graph $G_P=(V,E)$
where

\begin{itemize}
    \item $V$ is the set of nodes. Each node is one of either:
    \begin{enumerate*}
        \item a method entry point;
        \item a method exit point;
        \item an invocation of an method;
        \item a statement, for initializer, or a for update;
        \item a catch block;
        \item a finally block.
    \end{enumerate*}
    \item $E$ is the set of edges. Each edge is one of either:
    \begin{enumerate*}
        \item an intraprocedural edge;
        \item an interprocedural edge;
        \item an intraprocedural block entry edge;
        \item an intraprodudural block exit edge;
        \item an intraprocedural block entry and exit edge.
    \end{enumerate*}
\end{itemize}
\end{definition}

A block is a point where the control flow enters a new scope. We define five 
types of blocks:

\begin{labeling}{\textbf{The try, catch and finally blocks}}
    \item[\textbf{A basic block}] induced by a pair of brackets not following a method,
    conditional block, or try catch finally blocks.
    \item[\textbf{A conditional block}] induced by a conditional block, i.e. an if then 
    else statement or a loop structure.
    \item[\textbf{The try, catch and finally blocks}] induced by a try catch finally statement.
\end{labeling}

\subsection{The construction of the control flow graph}
The construction of the control flow graph is based on the control flow analysis
described in Principles of Program Analysis by Nielson et al. 
\cite{nielson2015principles}. It labels each statement in the program with an unique
number, and each statement has an initial label and final labels. We will describe 
the initial, the final labels, and the set of vertices and set of edges for each 
kind of compound statement below. 

We denote the initial node of a statement by $init : Statement \rightarrow Node$, the 
final nodes of a statement by $final : Statement \rightarrow \{Node\}$. 
We use $V(x)$ and $E(x)$ to denote the set of vertices and edges generated by object
$x$. We use $node(x)$, $edge(x)$, and $edges(x, y)$ to denote a new node, or new 
edge(s) for objects $x$ and $y$.

\subsubsection*{A constructor or method}
Given a constructor or method $M$ with body $S$, we will have:
\begin{align*}
    V &=\{node(M_{entry}), node(M_{exit})\} \cup V(S) \\ \\
    E &=
    \begin{cases}
      edge(node(M_{entry}), node(M_{exit})) 
        & \text{if } V(S) = \emptyset \\
      edge(node(M_{entry}), init(S)) \cup edges(final(S), node(M_{exit})) \cup E(S)
        & \text{otherwise}
    \end{cases}
\end{align*}

Or less formally, the vertices are the method entry point the method exit point, 
and the vertices created by the body of the method. The edges are an edge from
the method entry point to the initial of the body, and the edges from the final
of the body to the method exit point.

\subsubsection*{A sequence of statements}
Given a sequence $S$ of statements $S_1$ and $S_2$. We have that:
\begin{align*}
    init(S)  &= init(S_1) \\ \\
    final(S) &= final(S_2) \\ \\
    V &= V(S_1) \cup V(S_2) \\ \\
    E &= 
    \begin{cases}
        E(S_1) 
            & \\ \quad \text{if } S_1 \text{ is a \javainline{break} or \javainline{continue}} \\ \\
        edges(final(S_1), init(S_2)) \cup edges(breaks(S_1), init(S_2)) \cup E(S_1) \cup E(S_2) 
            & \\ \quad \text{if } S_1 \text{ is a loop} \\ \\
        edges(final(S_1), init(S_2))  \cup E(S_1) \cup E(S_2) 
            & \\ \quad \text{otherwise}
    \end{cases}
\end{align*}

Or less formally, the vertices are the vertices of the statements. The edges are
the edges of the statements, and we have three cases:
\begin{itemize}
    \item We have a \javainline{break} or \javainline{continue} statement: we
    end the sequence at the first statement;
    \item we have a loop statement: we add an edge from the final of the first 
    statement and the break statements of the loop to the initial of second 
    statement;
    \item we have any other statement: we add an edge from the final of the first
    statement to the initial of the next statement.
\end{itemize}

\subsubsection*{An if else statement}
Given an if else statement $I$, with the body of the true branch $S_\top$ and the
body of the false branch $S_\bot$, we will have:
\begin{align*}
    init(I)  &= node(I) \\ \\
    final(I) &= final(S_\top) \cup final(S_\bot) \\ \\
    V        &= \{node(I)\} \cup V(S_\top) \cup V(S_\bot) \\ \\
    E        &= edge(node(I), init(S_\top)) \cup edge(node(I), init(S_\bot)) 
                \cup E(S_\top) \cup E(S_\bot)
\end{align*}

Or less formally, the vertices are the if else statement itself, and the nodes
of the true branch and the false branch. The edges are, an edge from the 
if else statement itself to the initial of the body of the true branch and to
the initial of the body of the false branch.

\subsubsection*{A while loop}
Given a while loop $L$, with guard $g$ and body $S$, we have:
\begin{align*}
    init(L) &=
    \begin{cases}
      node(L) & \text{if } V(g) = \emptyset \\
      init(g) & otherwise
    \end{cases} \\ \\
    final(L) &= \{node(L)\} \\ \\
    V &= \{node(L)\} \cup V(S) \cup V(g) \\ \\
    E &= edge(node(L), init(S)) \cup edges(final(S), init(L))
       \\ & \quad \cup edges(continues(L), init(L)) \cup E(S) \cup E(g)
\end{align*}

where $continues(L)$ denotes the \javainline{continue} statements that belong
to the loop $L$.

Or less formally, the vertices are the loop itself, and the nodes of the loop 
body. The edges are, an edge from the loop itself to the initial body, edges from
the final of the body to the loop itself, edges from the continue statements of
this loop to the loop itself and the edges generated by the body of this loop.

Note that a while loop is always followed by a \javainline{CProver.assume(!g)}
statement.

\subsubsection*{A try catch finally statement}
Given a try statement $T$ with body $S_T$ catch statements $[C_1, \ldots, C_n]$
with bodies $[S_1, \ldots, S_n]$ and possibly a finally statement $F$ with body
$S_F$, we will have:
\begin{align*}
    init(T)  &= node(T) \\ \\
    final(T) &= 
    \begin{cases}
        final(S_F) & \text{if } T \text{ has a \javainline{finally} block}  \\
        final(S_n) & \text{otherwise}
    \end{cases} \\ \\
    V &= \{node(T), node(C_i), \ldots, node(C_n) \} \cup \{node(F)\} \cup V(S_T) \cup V(S_1) \cup \ldots \cup V(S_n) \cup S_F \\ \\
    E &= edge(node(T), init(S_T)) \cup edges(final(S_T), node(C_1)) \cup edge(node(C_1), init(S_1)) 
        \\ & \quad \cup edges(final(S_1), node(C_2)) \cup \ldots \cup edges(final(C_n), node(F))
        \\ & \quad  \cup edge(node(F), init(S_F)) \cup E(S_T) \cup E(S_1) \cup \ldots \cup E(S_n) \cup E(S_F) 
\end{align*}

Or less formally, the vertices are the try statement, catch statements and finally
statement with all bodies of these blocks combined. The edges are an edge from
the try statement to the init of the try statement itself, and we chain all
finals of the following blocks to the nodes of the try statements that follow.

\subsubsection*{A for loop}
Given a for loop $L$ with body $S$ and initialization expression $i$, guard $g$,
and update expression $u$.
\begin{align*}
    init(L) &= 
    \begin{cases}
        node(i) & \text{if }i \text{ exists} \\
        node(L) & \text{otherwise}
    \end{cases} \\ \\
    final(L) &= \{node(L)\} \\ \\
    V        &= \{node(i), node(L), node(u)\} \cup V(S) \\ \\ 
    E        &= edge(node(i), node(L)) \cup edge(node(L), init(S)) \cup edges(final(S), node(u))
                \\ & \quad \cup edges(node(u), node(L)) \cup edges(continues(L), node(u)) \cup E(S)
\end{align*}

Or less formally, we have a node for the loop itself, the initialization, the 
update and the body of the loop. We have an edge from the update to the loop
itself, an edge from the loop itself to the initial of the body. We have an edge from
the final of the body to the update, or the loop itself, depending if the update
exists. Finally, we have an edge from the update to the loop itself, if the update
exists.

Note that a for loop is always followed by a \javainline{CProver.assume(!g)}
statement.

\subsection{Reachability analysis}
The tool has a subphase in place which allows for reachability analysis. This
reachability analysis is a depth-first search starting at node $v \in V$ which
corresponds to the method entry point of the method to be verified. This subphase
is currently disabled as it may construct correct, but hard to read graphs, when
verifying a specific method that is not \javainline{main}.


\section{Linearization}
The third phase is the linearization phase; the phase that takes a control flow 
graph and generates all program paths from the starting point of the method to
be verified until the last point of method to be verified, up to the length $k$.

The verification starts at the method entry point node of the method to be verified
and ends at the method exit point of the method to verified if there are no more 
stack frames on the call stack. This is due to recursive method calls.

The length of a program paths is defined by the number of basic statements in the
program path, not counting any empty statement, i.e. \javainline{;}. This choice
is made as most realistic programs do not contain empty statements, but the tool
inserts skip statements at multiple places, keeping $k$ similar to the original 
program.

A program path is defined as a list of \haskellinline{PathStmt}, a 2-tuple 
containing a \haskellinline{PathType} and a \haskellinline{PathMetaInfo}. The 
\haskellinline{PathType} can be either a basic statement, an block entry or an 
block exit. The \haskellinline{PathMetaInfo} contains information about the package,
class and method, and the call name of the method this \haskellinline{PathType} is
enclosed in.

\subsection{The construction of the program paths}
The program path construction algorithm works as follows. We traverse edges of
the control flow graph from the method entry node of the method to be verified
until we either reach the final node, or the generated program path exceeds the
maximum length $k$.

During the traversal, we keep track of an accumulator \haskellinline{PathAccumulator} 
which is a 5-tuple containing:

\begin{itemize}
    \item The \haskellinline{CallHistory}: the number of times each method is called;
    \item the \haskellinline{StmtManipulations}: the method renaming that has to be
    performed at each node;
    \item the \haskellinline{CallStack}: the stack of method calls;
    \item the \haskellinline{ProgramPaths}: the program paths generated thus far;
    \item and the \haskellinline{Int}: the length we can append to the current
    paths, i.e. $k-$current length of the program paths generated thus far. 
\end{itemize}

When we reach a node that contains a statement, we append it to all program paths
and check if this node contains an method calls. If this is the case, we traverse
the expression in the statement and rename the method calls, this is done in the
\haskellinline{Linearization.Renaming} module.

When we traverse an edge, we check if the edge contains a block entry or exit. If 
this is the case, we add this block entry or exit to all paths and continue the
traversal. The complete algorithm can be found in the 
\haskellinline{Linearization.Path} module.

\section{Compilation}
The fourth phase is the compilation phase. In this file, the program paths are
reconstructed into valid Java programs. These Java programs are then compiled into
Java bytecode using \command{javac} and packed into jar files using \command{jar}.
This phase runs in parallel, speeding up the compilation. The number of worker
threads can be changed by tweaking a verification argument.

The compilation phase works by reconstructing each program path back into a 
\haskellinline{CompilationUnit'}, which is then compiled and packed. This
reconstruction works by grouping the program path by its classes, then by its 
methods and constructors, then by its blocks, and finally by its statements.

The compiler does not make many modifications to the program paths: the main
modification is that constructors, and constructor calls, are changed into 
static methods and regular method invocations. This modification is necessary as
constructors cannot have names that are not equal to the class name. We need
unique constructor names as same the same constructor can be called multiple times,
but each having an unique program sub-path associated with that constructor call.

\subsection{External filtering and modification of program paths}
The \haskellinline{Arguments} value given to the verification tool contains a
record value, shown in listing \ref{listing:path_filter}. This function allows 
for the adjustment and filtering of specific abstract syntax trees that are
created from the generated program paths.
\begin{Haskell}{listing:path_filter}{Filtering record value of \haskellinline{Arguments}.}
pathFilter :: CompilationUnit' -> Maybe CompilationUnit'
\end{Haskell}
When this function results in \haskellinline{Nothing}, the abstract syntax tree 
will be filtered. When this function results in \haskellinline{Just p}, the 
abstract syntax tree \haskellinline{p} will be verified.

\section{Verification}
The fifth and final phase is the verification phase. This phase runs JBMC on
each compiled Java program and parses the XML output of JBMC. Similar to the
compilation phase, this phase runs in parallel.

\chapter{How to use the tool} \label{chap:usage}
The tool can be used in two ways: using GHCi and by running the excutable. The tool
can be build using the command \command{make repl}, which also starts GHCi.

When GHCi is used, there are 3 functions which provide an easy to use interface
to verify: \haskellinline{verify}, \haskellinline{verifyWithMaximumDepth}, and
\haskellinline{verifyFunctionWithMaximumDepth}. There is an \haskellinline{arguments}
constant defined which allow for the tweaking of the arguments for the verification
process.

When running the executable, the arguments in listing \ref{listing:arguments} 
can be used to tweak the argument for the verification process. For example,
the command \command{bvj "examples/Test.java" -c -k10 -t4} verifies the file
\command{examples/Test.java} printing minimal information, with a maximum
program path length of 10 and uses 4 threads.

\begin{lstlisting}[basicstyle=\small\ttfamily, caption={Argument for the executable}, label={listing:arguments}]
Parameters:
File    Path to the file to be verified.

Flags:
-c           --compact                        Display less information.
-r           --remove                         Remove the output files.
-f[FUNCTION] --function[=FUNCTION]            Verify the function.
-k[DEPTH]    --depth[=DEPTH]                  Maximum program path generation depth.
-t[THREADS]  --threads[=THREADS]              Number of threads.
-u[UNWIND]   --unwind[=UNWIND]                Maximum loop unwinding in JBMC.
             --verification-depth[=VER-DEPTH] Maximum depth in JBMC.
\end{lstlisting}

\section{Prerequisites}
Using the tool requires the following:

\begin{enumerate}
    \item The Java development kit installed, ensure that \command{javac} and
    \command{jar} can be found in the PATH variable;
    \item the JBMC tool, ensure that that \command{jbmc} can be found in the PATH 
    variable;
    \item the following Haskell libraries:
    \begin{multicols}{3}
        \begin{itemize}
            \item \command{fgl 5.7.0.0}
            \item \command{directory 1.3.0.2}
            \item \command{terminal-progress-bar}
            \item \command{command}
            \item \command{filepath}
            \item \command{dates}
            \item \command{pretty}
            \item \command{xeno}
            \item \command{utf8-string}
            \item \command{language-java}
            \item \command{transformers}
            \item \command{containers}
            \item \command{parallel-io}
            \item \command{split}
            \item \command{simple-get-opt}
        \end{itemize}
    \end{multicols}
\end{enumerate}


\chapter{Conclusion} \label{chap:conclusion}
Concluding, we presented a bounded model checking tool which covers a subset of
the Java language. The examples in the appendix show that the tool can correctly 
detect errors in algorithms, like merge sort. Making this a feasible approach to
formally verify Java source code, instead of Java bytecode, which other bounded
model checking tools focus on.

\section{Possible future work}
\subsection*{Extend the subset of the Java language}
Currently, the tool only allows for the verification of a subset of the Java
language. This can be extended to allow for more Java programs to be verified.

One specific example is allowing \javainline{break} and \javainline{continue}
statements inside a \javainline{try}, \javainline{catch}, or \javainline{finally}
block inside a loop.

\subsection*{Add support for multi-threaded programs}
The developers of JBMC claim to have multi-threading support as one of their top
priorities. When JBMC supports multi-threading, the tool can be extended to support
multi-threading as well.

\bibliographystyle{plain}
\bibliography{bibliography.bib}

\begin{appendices}
    \chapter{Examples} \label{app:examples}

\section{The Fibonacci sequence}
Let us have a look at the program in listing \ref{listing:example_fibonacci}. This
program computes the second fibonacci number, and asserts if it is equal to 1. This
program generates the control flow graph shown in figure \ref{fig:cfg_fib}.

\begin{Java}{listing:example_fibonacci}{\text{Program verifying the fib method.}}
class Main {
    public static void main(String[] argv) {
        assert fib(2) == 1 : "fib(2) != 1";
    }

    public static int fib(int x) {
        if (x == 0) {
            return 0;
        } else if (x == 1) {
            return 1;
        } else {
            return fib(x - 1) + fib(x - 2);
        }
    }
}
\end{Java}

When we verify this program using $k=10$, we get the following five program 
paths, of which the shortest is:

\begin{lstlisting}[basicstyle=\small\ttfamily]
[CProver.assume(x==0); { return 0; } assert Main_fib0(2)==1 : "fib(2) != 1";]
\end{lstlisting}

The tool shows that all five program paths are correct. 

Modifying the \javainline{assert fib(2) == 1 : "fib(2) != 1";} to be
\javainline{assert fib(2) == 100 : "fib(2) != 100";} and running the tool again
shows that the assertion does not hold.

\begin{figure}[H]
    \caption{Control flow graph of listing \ref{listing:example_fibonacci}.}
    \label{fig:cfg_fib}
    \centering
    \begin{tikzpicture}[node/.style = {draw, circle}, node distance=1.23cm]
        \node(1)  [label=left:{entry of 'Main.main'}]        {1};
        \node(2)  [below=of 1,label=left:{;}]                           {2};
        \node(3)  [below=of 2,label=left:{call of 'Main.fib'}]          {3};
        \node(7)  [right=of 3,label={entry of 'Main.fib'}]         {7};
        \node(8)  [below=of 7,label=left:{;}]                           {8};
        \node(9)  [below=of 8,label=left:{;}]                           {9};
        \node(10) [below=of 9,label=left:{if (x == 0)}]                 {10};
        \node(11) [below left=of 10,label=left:{return 0;}]                   {11};
        \node(12) [below right=of 10,label=right:{;}]                           {12};
        \node(13) [below=of 12,label=left:{if (x == 1)}]                 {13};
        \node(14) [below left=of 13,xshift=-1cm,label=left:{return 1;}]                   {14};
        \node(15) [below right=of 13,label=left:{call of 'Main.fib'}]          {15};
        \node(16) [below=of 15,label=left:{call of 'Main.fib'}]          {16};
        \node(17) [below=of 16,label={return fib(x-1) + fib(x-2);}] {17};
        \node(18) [below=of 17,label=right:{;}]                           {18};
        \node(19) [below=of 18,label=right:{;}]                           {19};
        \node(20) [below=of 19,label=right:{;}]                           {20};
        \node(21) [below=of 20,label=below:{exit of 'Main.fib'}]          {21};
        \node(4)  [left=of 21,xshift=-0.25cm,label=above:{assert fib(2) == 1}]          {4};
        \node(5)  [left=of 4,xshift=-0.25cm,label=above:{;}]                           {5};
        \node(6)  [left=of 5,xshift=-0.25cm,label=above:{exit of 'Main.main}]          {6};

        \draw[->] (1) -- (2);
        \draw[->] (2) -- (3);
        \draw[->] (3) -- (7);
        \draw[->] (4) -- (5);
        \draw[->] (5) -- (6);
        \draw[->] (7) -- (8);
        \draw[->] (8) -- (9);
        \draw[->] (9) -- (10);
        \draw[->] (10) -- (11);
        \draw[->] (10) -- (12);
        \draw[->] (11.south) to [bend right=90] (19.west);
        \draw[->] (12) -- (13);
        \draw[->] (13) -- (14);
        \draw[->] (13) -- (15);
        \draw[->] (14.south) to [bend right=90] (18.west);
        \draw[->] (15.east) to [bend right] (7.east);
        \draw[->] (16.east) to [bend right] (7.east);
        \draw[->] (17) -- (18);
        \draw[->] (18) -- (19);
        \draw[->] (19) -- (20);
        \draw[->] (20) -- (21);
        \draw[->] (21) -- (4);
        \draw[->] (21.east) to [bend right=90] (16.east);
        \draw[->] (21.east) to [bend right] (17.east);
    \end{tikzpicture}
\end{figure}

\section{Merge sort}
Let us have a look at the program in listing \ref{listing:example_sort}. This 
program defined an array \javainline{elems} which is passed to the \javainline{sort}
method, which is an implementation of merge sort.

The tool correctly verifies the assertion that the array is sorted.

\begin{Java}{listing:example_sort}{Program verifying the sort method.}
class Main {
    public static void main(String[] argv) {
        int[] elems = new int[] { 2, 1 };
        sort(elems, 0, elems.length - 1);
        assert (elems[0] == 1) && (elems[1] == 2);
    }

    public static int partition(int[] array, int low, int high)
    {
        int pivot = array[high];
        int i = low - 1;
        int temp;

        for (int j = low; j < high; j++) {
            if (array[j] <= pivot)
            {
                i++;
                temp = array[i];
                array[i] = array[j];
                array[j] = temp;
            }
        }

        temp = array[i+1];
        array[i+1] = array[high];
        array[high] = temp;

        return i + 1;
    }

    public static void sort(int[] array, int low, int high) {
        if (low < high) {
            int pi = partition(array, low, high);
            sort(array, low, pi - 1);
            sort(array, pi + 1, high);
        }
    }
}
\end{Java}
\end{appendices}

\end{document}