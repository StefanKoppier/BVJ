\chapter{How to use the tool} \label{chap:usage}
The tool can be used in two ways: using GHCi and by running the excutable. The tool
can be build using the command \command{make repl}, which also starts GHCi.

When GHCi is used, there are 3 functions which provide an easy to use interface
to verify: \haskellinline{verify}, \haskellinline{verifyWithMaximumDepth}, and
\haskellinline{verifyFunctionWithMaximumDepth}. There is an \haskellinline{arguments}
constant defined which allow for the tweaking of the arguments for the verification
process.

When running the executable, the arguments in listing \ref{listing:arguments} 
can be used to tweak the argument for the verification process. For example,
the command \command{bvj "examples/Test.java" -c -k10 -t4} verifies the file
\command{examples/Test.java} printing minimal information, with a maximum
program path length of 10 and uses 4 threads.

\begin{lstlisting}[basicstyle=\small\ttfamily, caption={Argument for the executable}, label={listing:arguments}]
Parameters:
File    Path to the file to be verified.

Flags:
-c           --compact                        Display less information.
-r           --remove                         Remove the output files.
-f[FUNCTION] --function[=FUNCTION]            Verify the function.
-k[DEPTH]    --depth[=DEPTH]                  Maximum program path generation depth.
-t[THREADS]  --threads[=THREADS]              Number of threads.
-u[UNWIND]   --unwind[=UNWIND]                Maximum loop unwinding in JBMC.
             --verification-depth[=VER-DEPTH] Maximum depth in JBMC.
\end{lstlisting}

\section{Prerequisites}
Using the tool requires the following:

\begin{enumerate}
    \item The Java development kit installed, ensure that \command{javac} and
    \command{jar} can be found in the PATH variable;
    \item the JBMC tool, ensure that that \command{jbmc} can be found in the PATH 
    variable;
    \item the following Haskell libraries:
    \begin{multicols}{3}
        \begin{itemize}
            \item \command{fgl 5.7.0.0}
            \item \command{directory 1.3.0.2}
            \item \command{terminal-progress-bar}
            \item \command{command}
            \item \command{filepath}
            \item \command{dates}
            \item \command{pretty}
            \item \command{xeno}
            \item \command{utf8-string}
            \item \command{language-java}
            \item \command{transformers}
            \item \command{containers}
            \item \command{parallel-io}
            \item \command{split}
            \item \command{simple-get-opt}
        \end{itemize}
    \end{multicols}
\end{enumerate}
